\chapter{Data}
Data is collected for cryostat temperatures of $T\in\{ 2,4\}\ \si{\kelvin}$.
The \SI{2}{\kelvin} measurement features two series with $I_\text{sample}\in\{ 20,100\}\ \si{\micro\ampere}$, whereas the \SI{4}{\kelvin} measurement only features a series for $I_\text{sample}=\SI{20}{\micro\ampere}$.
The procedure for recording data is
\begin{enumerate}
	\item Adjust sample current.
	\item Record hall voltage while sweeping magnetic field up.
	\item Record longitudinal voltage while sweeping magnetic field down.
	\item Repeat steps 1-3
	\item Change temperature and repeat steps 2-4.
\end{enumerate}
A \texttt{PeakTech 12xx} digital oscilloscope is used to acquire voltage curves, which has to be replaced as soon as humanly possible.

\section{Methods} %TODO: Employed methods for data evaluation
\begin{figure}
	\centering
	\begin{subfigure}{.48\textwidth}
		\centering
		\includegraphics[width=\linewidth]{example-image-a}
		\caption{\textbf{Unfiltered data} The data before a savgol filter is applied. Part of data is not recoverable and thus not evaluable.}
	\end{subfigure}
	\hspace*{\fill}
	\begin{subfigure}{.48\textwidth}
		\centering
		\includegraphics[width=\linewidth]{example-image-b}
		\caption{\textbf{Filtered data} The data after filtering with increased STN ratio.}
	\end{subfigure}
	\caption[Comparison of (un)filtered example data]{\textbf{Comparison of (un)filtered example data}}
	\label{fig:comparison}
\end{figure}
\autoref{fig:comparison} shows a side-by-side comparison of example data prior to and after processing.
It is easy to see that part of the data is corrupted and cannot be evaluated as a result of using the \texttt{PeakTech} oscilloscope.
Therefore, the corrupted part is cut off and a Savitzky–Golay filter is applied to the data for the purpose of increasing the signal-to-noise ratio.

Since, in this setup, the magnetic field is recorded with the oscilloscope as well, the oscilloscope's measuremement and the magnetic field have to be put into context by calculating a scaling factor for the B-curve.

\section{Longitudinal Voltages} %TODO: Listing of long. voltages and their corresponding magnetic flux densities

\section{Hall Voltages} %TODO: Listing of hall voltages and their corresponding magnetic flux densities

\section{Identification of Plateaus} %TODO: Title says it all.

\section{Charge Carrier Density} %TODO: Dito.

\section{Fine-Structure Constant} %TODO: Why are you reading this?
