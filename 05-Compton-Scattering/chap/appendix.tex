\Appendix
\configureappendix

\section{Calculation of Differential Cross-Section}\label{appendix:cross}

To obtain the differential cross-section depicted in \autoref{fig:diff_cross}, first, the remaining gamma ray flux at the target has to be calculated since the activity of the source decays over time.
The lab manual specifies a gamma ray flux of $\phi_0 = \SI{1.54(9)e6}{\per\cm\squared\per\second}$ for the $^{137}$Cs source.
The reference measurement from the lab manual was conducted in July, 1971.
Since no day was specified, the middle of July is assumed.

Under these assumptions, the remaining gamma ray flux is
\begin{equation*}
	\phi_\text{rem} = \phi_0\cdot 0.5^{\left(\frac{t_\text{init}-t_\text{exp}}{t_{\nicefrac{1}{2}}}\right)} = \SI{5.18(30)e9}{\per\meter\squared\per\second},
\end{equation*}
where $t_\text{init}-t_\text{exp}$ denotes the timespan (in days) between the reference and the experiment conducted in this lab report.
$t_{\nicefrac{1}{2}}$ is the half-life of the $^{137}$Cs source in days.
The specified error is propagated like
\begin{equation*}
	\sigma_{\phi\text{, rem}} = \sigma_{\phi,0}\cdot\frac{\phi_\text{rem}}{\phi_0}.
\end{equation*}

The volume of the Al target is
\begin{equation*}
	V = \ell\cdot\pi\cdot\left(\frac{d}{2}\right)^2 = \SI{0.79(18)}{\cm\cubed},
\end{equation*}
where $\ell=\SI{1.0(1)}{\cm}$ and $d=\SI{1.0(1)}{\cm}$ denote the length and diameter of the target, respectively.
Errors on these quantities are propagated like
\begin{equation*}
	\sigma_V = \sigma_\ell\cdot\sqrt{\left(\frac{V}{\ell}\right)^2 + 4\left(\frac{V}{d}\right)^2},
\end{equation*}
since $\sigma_\ell=\sigma_d$.

Using the formula from the lab manual, the number of electrons is
\begin{equation*}
	n = \frac{N_\text{A}}{A}\cdot Z\cdot\rho\cdot V = \num{6.153(1376)e23},
\end{equation*}
where the error on $V$ is propagated via Gaussian error propagation (see above).

The solid angle correction regarding the scinitillation detector is
\begin{equation*}
	\Delta_\Omega = \frac{A}{d_\text{sci}^2} = \num{1.105e-2},
\end{equation*}
where $A=\pi\cdot\left(\SI{1.275}{\cm}\right)^2$ is the area of the scintillator and $d_\text{sci}$ is the distance between the target and the scintillation crystal.

Having calculated all of these quantities, we are now able to calculate the differential cross-section $\frac{\d\sigma}{\d\Omega}\left(\theta\right)$ for a given value of $\theta$
\begin{equation*}
	\frac{\d\sigma}{\d\Omega}\left(\theta\right) = \frac{R(\theta)}{\Delta_\Omega\cdot\phi_\text{rem}\cdot n}.
\end{equation*}
