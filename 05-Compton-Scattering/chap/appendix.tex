\Appendix
\configureappendix

\section{Calculation of Differential Cross-Section}\label{appendix:cross}

To obtain the differential cross-section depicted in \autoref{fig:diff_cross}, first, the remaining gamma ray flux at the target has to be calculated since the activity of the source decays over time.
The lab manual specifies a gamma ray flux of $\phi_0 = \SI{1.54(9)e6}{\per\cm\squared\per\second}$ for the $^{137}$Cs source.
The reference measurement from the lab manual was conducted in July, 1971.
Since no day was specified, the middle of July is assumed.

Under these assumptions, the remaining gamma ray flux is
\begin{equation*}
	\phi_\text{rem} = \phi_0\cdot\exp\left(\frac{t_\text{init}-t_\text{exp}}{t_{\nicefrac{1}{2}}}\right),
\end{equation*}
where $t_\text{init}-t_\text{exp}$ denotes the timespan (in days) between the reference and the experiment conducted in this lab report.
$t_{\nicefrac{1}{2}}$ is the half-life of the $^{137}$Cs source in days.
The specified error is propagated like
\begin{equation*}
	\sigma_{\phi\text{, rem}} = \sigma_{\phi,0}\cdot\exp\left(\frac{t_\text{init}-t_\text{exp}}{t_{\nicefrac{1}{2}}}\right).
\end{equation*}

The volume of the Al target is
\begin{equation*}
	V = \ell\cdot\pi\cdot\left(\frac{d}{2}\right)^2,
\end{equation*}
where $\ell=\SI{1.0(1)}{\cm}$ and $d=\SI{1.0(1)}{\cm}$ denote the length and diameter of the target, respectively.
