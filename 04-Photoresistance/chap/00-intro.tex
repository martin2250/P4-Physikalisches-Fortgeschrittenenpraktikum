\chapter{Theory}
\section{Bloch Wave}\label{sec:bloch}
\textbf{Theorem} (without proof) Let $V:\mathbb{R}^3\longrightarrow\mathbb{R}$ denote a scalar potential obeying periodic boundary conditions, such that $V(\mvec{r}) = V(\mvec{r}+s\cdot\mvec{T})$, where $s\in\mathbb{Z}$ and $\mvec{T}$ is the translation vector of a periodic lattice. Then, the stationary Schrödinger equation has solutions, which satisfy $\psi_k(\mvec{r}+\mvec{T}) = \e^{i\mvec{k}\cdot\mvec{T}}\cdot\psi_k(\mvec{r})$, where $\psi_k(\mvec{r})$ is a solution of the non-periodic stationary Schrödinger equation with a wave vector $\mvec{k}$.

\textbf{Remark} Such solutions are called \textit{Bloch waves}. The electron wave functions in a crystal lattice have a basis consisting entirely of Bloch wave energy eigenstates. This is easy to see, if one introduces a translation operator $\hat{T}$, such that $\hat{T}\psi (\mvec{r}) = \psi (\mvec{r}+\mvec{T})$. For a periodic potential, $\hat{T}$ commutes with the Hamiltonian, because the crystal has translational symmetry. Therefore, there is a simultaneous eigenbasis of the Hamiltonian and every possible translation operator (note that $\comm{T_i}{T_j} = 0$ for every $i\neq j$).
The eigenstates in this basis are energy eigenstates and Bloch waves at the same time.

\section{Electronic Band Structure}
If the electronic configuration inside of a periodic lattice is approximated as being free, according to the stationary Schrödinger equation for electrons, an electronic dispersion relation of
\begin{equation*}
	E(k) = \frac{\hbar^2 k_n^2}{2m},
\end{equation*}
is obtained, where the possible wave vectors $k_n = n\cdot\frac{\pi}{a}$ are determined by periodic boundary conditions.
$a$ denotes the lattice parameter.
This means that the dispersion relation for free electrons inside the lattice is periodically parabolic.

The periodicity can be translated into the reduced zone scheme by mirroring.
However, at the edges of the first Brillouin zone, the electrons satisfy the Bragg condition and are therefore mirrored to the opposite side of the first Brillouin zone.
This leads to the creation of standing waves in k-space at both zone edges.
For a periodic potential generated by the nuclei in the lattice, the electron probability density is able to snap into the lowest energy solutions in two different ways:
\begin{itemize}
\item \textbf{Maxima of $|\psi|^2$ at locations of nuclei:} As a consequence of the tight Coulomb binding between the electrons and nuclei, the energy is lowered with respect to the free parabolic dispersion. The parabola is bending downwards.
	\item \textbf{Maxima of $|\psi|^2$ between locations of nuclei:} Illustratively speaking, the electrons are located further away from the nuclei. Hence, the energy is raised with respect to the free parabolic dispersion as a consequence of the looser Coulomb binding. The parabola is bending upwards.
\end{itemize}

In total, the parabola bends cause the band structure by introducing a \textit{band gap}.
Electronic states which lie inside the band gap are forbidden.
The resulting bands are now filled up, considering the Pauli exclusion principle.
At \SI{0}{\kelvin}, there are no electrons with energies above the Fermi energy $E_\text{F}$.
At room temperature, however, some electrons are thermally excited into states above the Fermi energy, since the Fermi distribution
\begin{equation*}
  \overline{n_\lambda} = \frac{1}{\e^{\beta\left(\epsilon_\lambda-\mu\right)} + 1}
\end{equation*}
is only approaching a Heaviside theta function for $T\rightarrow 0$.

\section{Carrier Concentration in Non-Degenerate Semiconductors}\label{sec:cc}
Assuming that the Fermi energy of a semiconductor lies far within the band gap, Fermi-Dirac statistics may be replaced by Boltzmann statistics, effectively treating the electrons as a classical gas.
This approximation is called the \textit{non-degenerate approximation} and may only be used, if the Fermi energy is located far enough away from both band edges with respect to the thermal energy $k_\text{B}T$ of the system.

In the non-degenerate approximation, the carrier concentrations
\begin{align*}
	n &= \int_{E_\text{c}}^\infty\d E\ \nu_\text{n}(E)\cdot\e^{-\frac{E-E_\text{F}}{k_\text{B}T}} \\
	p &= \int_{-\infty}^{E_\text{v}}\d E\ \nu_\text{p}(E)\cdot\e^{-\frac{E_\text{F}-E}{k_\text{B}T}}
\end{align*}
can be calculated analytically and be used to calculate intrinsic carrier concentration as
\begin{alignat*}{1}
	n_0 &= \frac{2}{h^3}	\left(2 \uppi m_\text{e} k_\text{B} T \right)^{\nicefrac{3}{2}} \cdot	\e^{-\frac{E_\text{C} - E_\text{F}}{k_\text{B}T}},\\
	p_0 &= \frac{2}{h^3}	\left(2 \uppi m_\text{e} k_\text{B} T \right)^{\nicefrac{3}{2}} \cdot	\e^{-\frac{E_\text{V} - E_\text{F}}{k_\text{B}T}}.
\end{alignat*}
This is also called the equilibrium charge carrier density.

\section{Light Absorption}
Photons can only be absorbed when their energy is greater than the band gap of the semiconductor.
Thus, if the energy $\hbar \omega$ is smaller than the band gap $E_\text{G}$, the semiconductor is transparent for photons of that frequency.
If the energy is large enough, an electron is excited from the valence band into the conductance band, an electron-hole pair is formed.
Excess energy ($\hbar \omega > E_\text{G}$) is quickly dissapated as heat.

The electron-hole pairs have a mean lifetime $T_\text{l}$ after which they recombine, releasing the energy either as heat or light.

Charge carriers created by optical excitation are also called excess charge carriers.
Under normal circumstances, the number of excess charge carriers is small compared to the number of equilibrium charge carriers, so the total charge carrier concentrations are defined as $n = n_0 + \Delta n$ and $p = p_0 + \Delta p$.
Under the assumption that $\Delta n$ and $\Delta p$ are small enough not to affect the properties of all charge carriers, the total conductance changes linearly with the number of excess charge carriers.
The excess conducance is given by $\Delta \sigma = e \cdot \left( \Delta n \mu_\text{n} + \Delta p \mu_\text{p} \right)$

The amount of light absorbed in a thin layer orthogonal to the propagation direction $dI$ is proportional to the layer's thickness $dx$ and the intensity $I$ of the incident light $dI = - k I dx$.
A solid can be regarded as ininitely many, infinitesimally thin, such layers.
Solving the differential equation gives a decaying exponential relation between the remaining intensity of light and the penetration depth.
The absorption coefficient $k$ is highly dependent on the material and the wavelength of light.
The absorbed light generates charge carriers $\Delta n' = \Delta p' = \beta k I$, where $\beta$ denotes the quantum efficiency, the ratio of produced charge carriers and absorbed photons.
In a sufficiently thick solid the total absorbed intensity is independent of $k$, only the local charge carrier density is still affected.
