\chapter{Theory}
\section{Bloch Wave}\label{sec:bloch}
\textbf{Theorem} (without proof) Let $V:\mathbb{R}^3\longrightarrow\mathbb{R}$ denote a scalar potential obeying periodic boundary conditions, such that $V(\mvec{r}) = V(\mvec{r}+s\cdot\mvec{T})$, where $s\in\mathbb{Z}$ and $\mvec{T}$ is the translation vector of a periodic lattice. Then, the stationary Schrödinger equation has solutions, which satisfy $\psi_k(\mvec{r}+\mvec{T}) = \e^{i\mvec{k}\cdot\mvec{T}}\cdot\psi_k(\mvec{r})$, where $\psi_k(\mvec{r})$ is a solution of the non-periodic stationary Schrödinger equation with a wave vector $\mvec{k}$.

\textbf{Remark} Such solutions are called \textit{Bloch waves}. The electron wave functions in a crystal lattice have a basis consisting entirely of Bloch wave energy eigenstates. This is easy to see, if one introduces a translation operator $\hat{T}$, such that $\hat{T}\psi (\mvec{r}) = \psi (\mvec{r}+\mvec{T})$. For a periodic potential, $\hat{T}$ commutes with the Hamiltonian, because the crystal has translational symmetry. Therefore, there is a simultaneous eigenbasis of the Hamiltonian and every possible translation operator (note that $\comm{T_i}{T_j} = 0$ for every $i\neq j$).
The eigenstates in this basis are energy eigenstates and Bloch waves at the same time.

\section{Electronic Band Structure}
If the electronic configuration inside of a periodic lattice is approximated as being free, according to the stationary Schrödinger equation for electrons, an electronic dispersion relation of
\begin{equation*}
	E(k) = \frac{\hbar^2 k_n^2}{2m},
\end{equation*}
is obtained, where the possible wave vectors $k_n = n\cdot\frac{\pi}{a}$ are determined by periodic boundary conditions.
$a$ denotes the lattice parameter.
This means that the dispersion relation for free electrons inside the lattice is periodically parabolic.

The periodicity can be translated into the reduced zone scheme by mirroring.
However, at the edges of the first Brillouin zone, the electrons satisfy the Bragg condition and are therefore mirrored to the opposite side of the first Brillouin zone.
This leads to the creation of standing waves in k-space at both zone edges.
For a periodic potential generated by the nuclei in the lattice, the electron probability density is able to snap into the lowest energy solutions in two different ways:
\begin{itemize}
\item \textbf{Maxima of $|\psi|^2$ at locations of nuclei:} As a consequence of the tight Coulomb binding between the electrons and nuclei, the energy is lowered with respect to the free parabolic dispersion. The parabola is bending downwards.
	\item \textbf{Maxima of $|\psi|^2$ between locations of nuclei:} Illustratively speaking, the electrons are located further away from the nuclei. Hence, the energy is raised with respect to the free parabolic dispersion as a consequence of the looser Coulomb binding. The parabola is bending upwards.
\end{itemize}

In total, the parabola bends cause the band structure by introducing a \textit{band gap}.
Electronic states which lie inside the band gap are forbidden.
The resulting bands are now filled up, considering the Pauli exclusion principle.
At \SI{0}{\kelvin}, there are no electrons with energies above the Fermi energy $E_\text{F}$.
At room temperature, however, some electrons are thermally excited into states above the Fermi energy, since the Fermi distribution
\begin{equation*}
  \overline{n_\lambda} = \frac{1}{\e^{\beta\left(\epsilon_\lambda-\mu\right)} + 1}
\end{equation*}
is only approaching a Heaviside theta function for $T\rightarrow 0$.

\section{Carrier Concentration in Non-Degenerate Semiconductors}\label{sec:cc}
Assuming that the Fermi energy of a semiconductor lies far within the band gap, Fermi-Dirac statistics may be replaced by Boltzmann statistics, effectively treating the electrons as a classical gas.
This approximation is called the \textit{non-degenerate approximation} and may only be used, if the Fermi energy is located far enough away from both band edges with respect to the thermal energy $k_\text{B}T$ of the system.

In the non-degenerate approximation, the carrier concentrations
\begin{align*}
	n &= \int_{E_\text{c}}^\infty\d E\ \nu_\text{n}(E)\cdot\e^{-\frac{E-E_\text{F}}{k_\text{B}T}} \\
	p &= \int_{-\infty}^{E_\text{v}}\d E\ \nu_\text{p}(E)\cdot\e^{-\frac{E_\text{F}-E}{k_\text{B}T}}
\end{align*}
can be calculated analytically and be used to calculate intrinsic carrier concentration as
\begin{alignat*}{1}
	n_0 &= \frac{2}{h^3}	\left(2 \uppi m_\text{e} k_\text{B} T \right)^{\nicefrac{3}{2}} \cdot	\e^{-\frac{E_\text{C} - E_\text{F}}{k_\text{B}T}},\\
	p_0 &= \frac{2}{h^3}	\left(2 \uppi m_\text{e} k_\text{B} T \right)^{\nicefrac{3}{2}} \cdot	\e^{-\frac{E_\text{V} - E_\text{F}}{k_\text{B}T}}.
\end{alignat*}
This is also called the equilibrium charge carrier density.

\section{Light Absorption}\label{sec:theory:absorption}
Photons can only be absorbed when their energy is greater than the band gap of the semiconductor.
Thus, if the energy $\hbar \omega$ is smaller than the band gap $E_\text{G}$, the semiconductor is transparent for photons of that frequency.
If the energy is large enough, an electron is excited from the valence band into the conductance band, an electron-hole pair is formed.
Excess energy ($\hbar \omega > E_\text{G}$) is quickly dissipated as heat.

The electron-hole pairs have a mean lifetime $\tau$ after which they recombine, releasing the energy either as heat or light.

Charge carriers created by optical excitation are also called excess charge carriers.
Under normal circumstances, the number of excess charge carriers is small compared to the number of equilibrium charge carriers, so the total charge carrier concentrations are defined as $n = n_0 + \Delta n$ and $p = p_0 + \Delta p$.
Under the assumption that $\Delta n$ and $\Delta p$ are small enough not to affect the properties of all charge carriers, the total conductance changes linearly with the number of excess charge carriers.
The excess conductance is given by $\Delta \sigma = e \cdot \left( \Delta n \mu_\text{n} + \Delta p \mu_\text{p} \right)$

The amount of light absorbed in a thin layer orthogonal to the propagation direction $dI$ is proportional to the layer's thickness $dx$ and the intensity $I$ of the incident light $dI = - k I dx$.
A solid can be regarded as the limit of infinitely many, infinitesimally thin, layers.
Solving the differential equation gives a decaying exponential relation between the remaining intensity of light and the penetration depth.
The absorption coefficient $k$ is highly dependent on the material and the wavelength of light.
The absorbed light generates charge carriers $\Delta n' = \Delta p' = \beta k I$, where $\beta$ denotes the quantum efficiency, the ratio of produced charge carriers and absorbed photons.
The quantum efficiency is usually smaller than one, $\beta > 1$ is only possible when charge carriers can obtain enough momentum to excite further electrons from the valence band.
In a sufficiently thick solid the total absorbed intensity is independent of $k$, only the local charge carrier density is still affected.

If there were no recombination processes, the charge carrier density would rise indefinitely.
In reality the charge carrier concentration quickly approaches a steady state, where the rates of generation and recombination of charge carriers are equal.
The semiconductor's properties in this steady state are denoted by $\Delta n_\text{st}$, $\Delta p_\text{st}$ and $\Delta \sigma_\text{st}$ and it holds
\begin{align*}
	\Delta n_\text{st} &= \Delta n' \cdot \tau_\text{n} \\
	\Delta p_\text{st} &= \Delta p' \cdot \tau_\text{p},
\end{align*}
where $\tau_i$ denotes the mean lifetime of the charge carriers.
This linear dependence is only valid for weak light intensity and is dominated by nonlinear effects of recombination at higher intensities.

\section{Linear Recombination}
The total rate of change of the electron density as discussed in \autoref{sec:theory:absorption} is given by
\begin{equation*}
	\Delta n' = \beta k I - \frac{\Delta n}{\tau_\text{n}}.
\end{equation*}
Solving this differential equation for $I(t) = I_0 \Theta (t)$ and $\Delta n (t = 0) = 0$, where $\Theta$ is the Heaviside step function, yields
\begin{equation}
	\Delta n (t) = \tau \beta k I_0 \left(1 - \e^{\nicefrac{-t}{\tau}}\right).
\end{equation}
This linear behavior is also observed for the excess conductance $\Delta \sigma$.

\section{Quadratic Recombination}\label{sec:quad-recomb}
If the equilibrium charge carrier density is negligably small in comparison to the excess charge carrier concentration, quadratic recombination of charge carriers is observed.
The differential equation changes to
\begin{equation*}
	\Delta n' = \beta k I - \gamma (\Delta n)^2,
\end{equation*}
which is satisfied by
\begin{equation*}
	\Delta n (t) = \sqrt{\frac{\beta k I_0}{\gamma}} \tanh \left(t \sqrt{\gamma \beta k I_0} \right)
\end{equation*}
for $I(t) = I_0 \Theta (t)$ and
\begin{equation*}
	\Delta n (t) = \sqrt{\frac{\beta k I_0}{\gamma}} \left(1 + t \sqrt{\gamma \beta k I_0} \right)^{-1}
\end{equation*}
for $I(t) = I_0 \Theta (-t)$

The stationary excess charge carrier density and conductance are both proportional to $\sqrt{I}$.
Also the mean lifetime of charge carries is no longer fixed, but is dependent on the instantaneous charge carrier density.

\section{Frequency Dependence}
The electron concentration's $n_\text{e}$ change with time can be derived by considering a continuity equation
\begin{equation}\label{eq:cont_el}
	\partial_tn_\text{e} = G_\text{e} - R_\text{e},
\end{equation}
where $G_\text{e}$ and $R_\text{e}$ denote the rates of generation and recombination, respectively.

If one modulates a constant illumination intensity with a sinusoidal oscillation, the rate of generation $G_\text{e}$ is given by
\begin{equation}\label{eq:genrate}
	G_\text{e}(t) = G_0 + \Delta G\cdot\e^{\iu\omega t},
\end{equation}
$G_0$ denoting a constant rate of generation in equilibrium, $\Delta G$ denoting the amplitude of modulation of $G$, $\iu$ denoting the imaginary unit, $\omega$ denoting the angular frequency of the modulation and $t$ denoting time.

Since the rate of recombination, in general, is nonlinear in $n_\text{e}$, we have to resign ourselves to a linear expansion
\begin{equation}\label{eq:rerate}
	R_\text{e}(t)\approx \frac{n_0}{\tau_0} + \frac{\Delta n(t)}{\tau_\text{e}},
\end{equation}
where $\Delta n(t)$ denotes the modulation of $n_\text{e}$ resulting from modulation of illumination according to
\begin{equation}\label{eq:ansatz_n}
	n_\text{e}(t) = n_0 + \Delta n(t).
\end{equation}
Such an approximation can be justified by having chosen a sufficiently small modulation amplitude.

Using \autoref{eq:ansatz_n} as an \textit{ansatz} for \autoref{eq:cont_el} and utilizing Equations \ref{eq:genrate} and \ref{eq:rerate}, yields a complex quantity
\begin{align}
	\Delta n(t) &= \frac{\Delta G\cdot\tau_\text{e}}{1+\omega^2\tau_\text{e}^2}\cdot\left(1 -\iu\omega\tau_\text{e}\right)\e^{\iu\omega t} \nonumber \\
	\Rightarrow |\Delta n(t)| &= \frac{\Delta G\cdot\tau_\text{e}}{\sqrt{1+\omega^2\tau_\text{e}^2}}.\label{eq:freq-dependence}
\end{align}

Plotting $|\Delta n(t)|$ against the frequency allows for determination of the mean electron lifetime $\tau_\text{e}$.
