\chapter{Data and Procedure}

\section{Setup}
This experiment pursues to characterize a CdS semiconductor as a photoresistor.
Light from a polychromatic light source is focused on the photoresistor using a lens system.
Discrete and graduated color filters can be employed to illuminate the photoresistor with light of a specific wavelength only.
Polarizing filters are used to vary the level of illumination.
To measure the frequency dependence of photoconductivity, the polychromatic light source is replaced by a blue luminescence diode.
Its intensity is modulated sinusoidally with variable frequency.

\section{Data}

\subsection{I-V Characteristic Curves}
\begin{figure}
	\centering
	\begin{subfigure}{0.45\textwidth}
		\centering
		\includegraphics[width=\textwidth]{./data/plots/IV-549.pdf}
		\captiond{I-V curve, $\lambda=\SI{549}{\nm}$}{}
		\label{fig:iv-549}
	\end{subfigure}
	\begin{subfigure}{0.45\textwidth}
		\centering
		\includegraphics[width=\textwidth]{./data/plots/IV-647.pdf}
		\captiond{I-V curve, $\lambda=\SI{647}{\nm}$}{}
		\label{fig:iv-647}
	\end{subfigure}
  \par\bigskip
	\begin{subfigure}{0.7\textwidth}
		\centering
		\includegraphics[width=\textwidth]{./data/plots/IV-all.pdf}
		\captiond{Overview over I-V curves}{with additional lab background and darkened photoresistor curves}
		\label{fig:iv-all}
	\end{subfigure}
	\captiond{I-V characteristic curves for various wavelengths}{including background measurements}
  \label{fig:iv}
\end{figure}

\autoref{fig:iv} shows the recorded I-V characteristics for light wavelengths of \SI{549}{\nm} and \SI{647}{\nm}, as well as lab background/photoresistor noise measurements caused by either additional illumination of the photoresistor with stray lightrays in the setup (which do not originate from the intended light source) or by electronic noise.

It is easy to see that the background is negligably small.
The observed curves show the intended linear behavior (see Subfigures \ref{fig:iv-549} and \ref{fig:iv-647}).
The photoresistor's resistance clearly changes with the wavelength of incident light.
Performing linear regression yields resistances
\begin{align*}
  R(\lambda = \SI{549}{\nm}) = \SI{718}{\ohm} \\
  R(\lambda = \SI{647}{\nm}) = \SI{217}{\ohm}.
\end{align*}

\subsection{Illuminance Dependency}
\begin{figure}
	\centering
	\includegraphics[width=.6\textwidth]{./data/plots/intensity-current.pdf}
	\captiond{Photocurrent over Illuminance}{}
	\label{fig:intense-current}
\end{figure}

\autoref{fig:intense-current} shows the measured photocurrent over illuminance.
Since the photocurrent is clearly proportional to the square root of illuminance, quadratic recombination of excess charge carriers is observed (as outlined in \autoref{sec:quad-recomb}).
We therefore conclude, that the concentration of equilibrium charge carriers is negligably small compared to excess carriers.

\subsection{Light Wavelength Dependency}

\subsection{Modulation Frequency Dependence}
The Photoresistor is illuminated by an LED, which is driven by a constant current with a shallow sinusoidal modulation superimposed.
This results in a very much not-sinusoidal current through the photoresistor.
A lock-in amplifier is used to measure the amplitude of the fundamental component of the photocurrent and also the phase with respect to the LED drive signal.

The amplitude of the fundamental is measured for multiple frequencies and shown in \autoref{fig:freq-current}.
Theoretically, the amplitude should be proportional to $|\Delta n(t)|$ (\autoref{eq:freq-dependence}).
This model does not fit the aquired data very well, so the corner frequency is determined by fitting a horizontal line to the points in log-log-space left of the corner frequency and a linear slope to the points right of the transition region.
The frequency where these lines intersect is $f_\text{i} = \SI{171}{\hertz}$ and corresponds the inverse of the mean charge carrier lifetime $\tau_\text{e} = \frac{1}{f_\text{i}} = \SI{5.8}{\milli\second}$.
% todo: adjust cutoff indices
\begin{figure}
	\centering
	\includegraphics[width=.6\textwidth]{./data/plots/frequency-current.pdf}
	\captiond{Photocurrent over Frequency}{(RMS of fundamental component)}
	\label{fig:freq-current}
\end{figure}
