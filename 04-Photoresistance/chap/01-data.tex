\chapter{Data and Procedure}

\section{Setup}

\section{Data}

\subsection{I-V Characteristic Curves}
\begin{figure}
	\centering
	\begin{subfigure}{0.45\textwidth}
		\centering
		\includegraphics[width=\textwidth]{./data/plots/IV-549.pdf}
		\captiond{I-V curve, $\lambda=\SI{549}{\nm}$}{}
		\label{fig:iv-549}
	\end{subfigure}
	\begin{subfigure}{0.45\textwidth}
		\centering
		\includegraphics[width=\textwidth]{./data/plots/IV-647.pdf}
		\captiond{I-V curve, $\lambda=\SI{647}{\nm}$}{}
		\label{fig:iv-647}
	\end{subfigure}
  \par\bigskip
	\begin{subfigure}{0.7\textwidth}
		\centering
		\includegraphics[width=\textwidth]{./data/plots/IV-all.pdf}
		\captiond{Overview over I-V curves}{with additional lab background and darkened photoresistor curves}
		\label{fig:iv-all}
	\end{subfigure}
	\captiond{I-V characteristic curves for various wavelengths}{including background measurements}
  \label{fig:iv}
\end{figure}

\autoref{fig:iv} shows the recorded I-V characteristics for light wavelengths of \SI{549}{\nm} and \SI{647}{\nm}, as well as lab background/photoresistor noise measurements caused by either additional illumination of the photoresistor with stray lightrays in the setup (which do not originate from the intended light source) or by electronic noise.

It is easy to see that the background is negligably small.
The observed curves show the intended linear behavior (see Subfigures \ref{fig:iv-549} and \ref{fig:iv-647}).
The photoresistor's resistance clearly changes with the wavelength of incident light.
Performing linear regression yields resistances
\begin{align*}
  R(\lambda = \SI{549}{\nm}) = \SI{718}{\ohm} \\
  R(\lambda = \SI{647}{\nm}) = \SI{217}{\ohm}.
\end{align*}

\subsection{Light Wavelength Dependency}

\subsection{Modulation Frequency Dependence}
