\chapter{Analysis of Data}

\section{Sample A}

\section{Sample B}

\section{Comparison of Hall Mobilities}
\begin{figure}
  \centering
  \includegraphics[width=.6\textwidth]{./data/plots/mob-comp.pdf}
  \captiond{Comparison of hall mobility for both samples}{}
  \label{fig:mob-comp}
\end{figure}

Hall mobility dependences for both samples are compared in \autoref{fig:mob-comp}.
It is easy to see that the curves exhibit different temperature dependencies for low temperatures.
Inside the inspected temperature region, the carrier mobility of GaAs (2DEG) rises, whereas, for the Ge-sample, it reaches a maximum before dropping for decreasing temperatures.

The GaAs-sample clearly is examined inside its polar mobility regime. Its hall mobility is expected to reach a maximum before decreasing like $T^{-\nicefrac{3}{2}}$ for even lower temperatures, as outlined in \autoref{subsec:phonons}. \todo{Check that, I have no idea, what I'm talking about.}
