\chapter{Analysis of Data}

\section{Sample A}

\section{Conductivity and Hall Coefficient}
For each temperature, four measurements are recorded: the current through the sample $I$, the longitudinal voltage $U_\text{L}$ with no external magnetic field and the hall voltage $U_\text{H}^{+,-}$ at a magnetic field of $\pm B = \pm \SI{0.5}{\tesla}$.

The conductivity $\sigma$ of the sample is calculated as
\begin{equation*}
	\sigma = \frac{I}{U_\text{L}} \frac{l}{w h}
\end{equation*}
and the Hall coefficient $R_\text{H}$ is calculated as
\begin{equation*}
	R_\text{H} = \frac{U_\text{H}^+ - U_\text{H}^-}{2 I} \frac{1}{h B},
\end{equation*}
where $l$ denotes the sample's length along the current flow, $w$ the distance between the hall voltage contacts and $h$ denotes the other dimension.

A plot of $\frac{1}{\left|R_\text{H}\right|}$ and $sigma$ over $T$ is shown in \autoref{fig:sample-a-conductivity}.
The transition to the intrinsic region is clearly visible and it is estimated to start at \SI{288}{\kelvin}.
The end of the extrinsic region is better observed in a log-log-plot of the product $\sigma \cdot \left| R_\text{H} \right|$, shown in \autoref{fig:sample-a-product}.
It is estimated to end at \SI{160}{\kelvin}.


\begin{figure}
  \centering
  \includegraphics[width=.6\textwidth]{./data/plots/sample-a-conductivity.pdf}
  \captiond{Conductivity and Hall coefficient over temperature of sample A}{}
  \label{fig:sample-a-conductivity}
\end{figure}

\begin{figure}
  \centering
  \includegraphics[width=.6\textwidth]{./data/plots/sample-a-product.pdf}
  \captiond{Product of conductivity and Hall coefficient over temperature of sample A}{}
  \label{fig:sample-a-product}
\end{figure}

\section{Sample B}

\section{Comparison of Hall Mobilities}
\begin{figure}
  \centering
  \includegraphics[width=.6\textwidth]{./data/plots/mob-comp.pdf}
  \captiond{Comparison of hall mobility for both samples}{}
  \label{fig:mob-comp}
\end{figure}

Hall mobility dependences for both samples are compared in \autoref{fig:mob-comp}.
Apart from their obviously different absolute values, it is easy to see that the curves exhibit different temperature dependencies for low temperatures (with respect to the observed temperature range).

\textbf{The germanium sample (A)} shows a temperature dependence which is in accordance with the theory outlined in \autoref{subsec:phonons}: For high temperatures, scattering at acoustic phonons makes a major contribution to the hall mobility $\left(T^{-\nicefrac{3}{2}}\right)$.
For low temperatures, scattering at charged defects results in a $\propto T^{\nicefrac{3}{2}}$ behavior.

\textbf{The GaAs composite sample (B)} appears to show the expected behavior as well.
For this temperature regime, scattering at optical phonons should make the main contribution to the overall behavior so that a behavior like $\mu\propto \e^\frac{\hbar \omega}{k T} - 1$ is expected.
This assumption is confirmed by the observed data.

% todo: move this to appendix and uncomment
% \begin{table}
% 	\caption{title}
% 	\centering
% 	\input{data/plots/table-a.agtex}
% \end{table}
