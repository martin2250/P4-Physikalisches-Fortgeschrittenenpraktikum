\chapter{Analysis of Data}

\section{Sample A}

\section{Conductivity and Hall Coefficient}
\begin{figure}
  \centering
  \includegraphics[width=.6\textwidth]{./data/plots/sample-a-conductivity.pdf}
  \captiond{Conductivity and Hall coefficient over temperature of sample A}{}
  \label{fig:sample-a-conductivity}
\end{figure}

\begin{figure}
  \centering
  \includegraphics[width=.6\textwidth]{./data/plots/sample-a-product.pdf}
  \captiond{Product of conductivity and Hall coefficient over temperature of sample A}{}
  \label{fig:sample-a-product}
\end{figure}

The conductivity $\sigma$ of the sample is calculated as
\begin{equation*}
	\sigma = \frac{I}{U_\text{L}} \frac{l}{w h}
\end{equation*}
and the Hall coefficient $R_\text{H}$ is calculated as
\begin{equation*}
	R_\text{H} = \frac{U_\text{H}^+ - U_\text{H}^-}{2 I} \frac{1}{h B},
\end{equation*}
where $l$ denotes the sample's length along the current flow, $w$ the distance between the hall voltage contacts and $h$ denotes the other dimension.

A plot of $\frac{1}{\left|R_\text{H}\right|}$ and $\sigma$ over $T$ is shown in \autoref{fig:sample-a-conductivity}.
The transition to the intrinsic region is clearly visible and it is estimated to start at \SI{288}{\kelvin}.
The end of the extrinsic region is better observed in a log-log-plot of the product $\sigma \cdot \left| R_\text{H} \right|$, shown in \autoref{fig:sample-a-product}.
It is estimated to end at \SI{160}{\kelvin}.

\section{Conduction Type}
From the sign of xxx and the geometry of the setup, shown in \autoref{fig:todo}, \todo{}

\section{Intrinsic Charge Carrier Density}
\begin{figure}
	\centering
	\includegraphics[width=.5\textwidth]{./data/plots/n.pdf}
	\captiond{Intrinsic carrier concentration as a function of temperature}{}
	\label{fig:intr_carrier}
\end{figure}

Rearranging Equation (55) from the experiment manual yields the following relation between the charge carrier density $n_\text{i}$ and the hall coefficient $R_\text{H}$:
\begin{equation*}
	n_\text{i} \left(R_\text{H}\left( T \right), T \right) = \frac{1}{e \cdot R_\text{H}\left(T\right)} \frac{ 1 - b \left( T \right) }{ 1 + b \left( T \right) },
\end{equation*}
where the ratio of positive and negative charge carriers' mobilities
\begin{equation*}
	b \left( T \right) = \frac{\mu_\text{n}}{\mu_\text{p}} = \num{1.24553} + \num{0.00107} \cdot T,
\end{equation*}
is calculated using the empirical formula given in the experiment manual.

\autoref{fig:intr_carrier} shows the intrinsic carrier concentration $n_\text{i}$ as a function of temperature $T$.
The measured data obviously reflects the theoretical model, since the dependence is expected to be $\propto T^{\nicefrac{3}{2}}\cdot\e^{-\frac{E_\text{G}}{2k_\text{B}T}}$ as outlined in \autoref{sec:cc} (note that $n_\text{i}(T)$ is scaled logarithmically on the y-axis).

The value of $n_\text{i}\left(T = \SI{300}{\kelvin}\right) = \SI{2.316e19}{\per\cubic\meter}$ is obtained by a linear interpolation between the closest measured values and not, despite the experiment manual's instructions, by using values for the y-intercept and $E_\text{g,0}$ from the Arrhenius plot (discussed in \autoref{sec:arrhenius}).
The Arrhenius plot has been obtained by calculating $n_\text{i}(T)$ in the first place, so the best approach to obtain the intrinsic carrier concentration at \SI{300}{\kelvin} is using the original $n_\text{i}(T)$ dependence.

\section{Arrhenius Plot}\label{sec:arrhenius}
\begin{figure}
	\centering
	\includegraphics[width=.5\textwidth]{./data/plots/arrhenius.pdf}
	\captiond{Arrhenius plot for sample A}{The expected linear dependence is easy to be seen.}
	\label{fig:arrhenius}
\end{figure}

\autoref{fig:arrhenius} shows an Arrhenius plot for sample A (discussed at the end of \autoref{sec:cc}).
As indicated by an $R^2$-value of $0.9998\approx 1$, a good linear dependence is observed for the underlying data, confirming the validness of the theoretical model.
By identifying the parameters of the performed fit, values
\begin{align*}
	E_\text{g,0} = \SI{0.774}{\eV} \\
	\text{y-intercept} = \num{50.993} % whatever units
\end{align*}
are acquired.

Using the computed value for $E_\text{g,0}$ and the relation $E_\text{g}(T) = E_\text{g,0} -\upalpha T$ for the band gap energy, where $\upalpha=\SI{4e-4}{\eV\per\kelvin}$ is specified in the experiment manual, the band gap energy at \SI{300}{\kelvin}
\begin{equation*}
	E_\text{g}\left(T = \SI{300}{\kelvin}\right) = \SI{0.654}{\eV}
\end{equation*}
is obtained.

\section{Sample B}

\section{Comparison of Hall Mobilities}
\begin{figure}
  \centering
  \includegraphics[width=.6\textwidth]{./data/plots/mob-comp.pdf}
  \captiond{Comparison of hall mobility for both samples}{}
  \label{fig:mob-comp}
\end{figure}

Hall mobility dependences for both samples are compared in \autoref{fig:mob-comp}.
Apart from their obviously different absolute values, it is easy to see that the curves exhibit different temperature dependencies for low temperatures (with respect to the observed temperature range).

\textbf{The germanium sample (A)} shows a temperature dependence which is in accordance with the theory outlined in \autoref{subsec:phonons}: For high temperatures, scattering at acoustic phonons makes a major contribution to the hall mobility $\left(\propto T^{-\nicefrac{3}{2}}\right)$.
For low temperatures, scattering at charged defects results in a $\propto T^{\nicefrac{3}{2}}$ behavior.

\textbf{The GaAs composite sample (B)} appears to show the expected behavior as well.
For this temperature regime, scattering at optical phonons should make the main contribution to the overall behavior so that a behavior like $\mu\propto \e^\frac{\hbar \omega}{k T} - 1$ is expected.
This assumption is confirmed by the observed data.

% todo: move this to appendix and uncomment
% \begin{table}
% 	\caption{title}
% 	\centering
% 	\input{data/plots/table-a.agtex}
% \end{table}
