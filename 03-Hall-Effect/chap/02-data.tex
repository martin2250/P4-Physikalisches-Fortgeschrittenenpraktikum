\chapter{Analysis of Data}

\section{Sample A}

\section{Sample B}

\section{Comparison of Hall Mobilities}
\begin{figure}
  \centering
  \includegraphics[width=.6\textwidth]{./data/plots/mob-comp.pdf}
  \captiond{Comparison of hall mobility for both samples}{}
  \label{fig:mob-comp}
\end{figure}

Hall mobility dependences for both samples are compared in \autoref{fig:mob-comp}.
Apart from their obviously different absolute values, it is easy to see that the curves exhibit different temperature dependencies for low temperatures (with respect to the observed temperature range).

\textbf{The germanium sample (A)} shows a temperature dependence which is in accordance with the theory outlined in \autoref{subsec:phonons}: For high temperatures, scattering at acoustic phonons makes a major contribution to the hall mobility $\left(T^{-\nicefrac{3}{2}}\right)$.
For low temperatures, scattering at charged defects results in a $\propto T^{\nicefrac{3}{2}}$ behavior.

\textbf{The GaAs composite sample (B)} appears to show the expected behavior as well.
For this temperature regime, scattering at optical phonons should make the main contribution to the overall behavior so that a behavior like $\mu\propto \e^\frac{\hbar \omega}{k T} - 1$ is expected.
This assumption is confirmed by the observed data.
