\chapter{Theory}
\section{Electronic Band Structure}
If the electronic configuration inside of a periodic lattice is approximated as being free, according to the stationary Schrödinger equation for electrons, an electronic dispersion relation of
\begin{equation*}
	E(k) = \frac{\hbar^2 k_n^2}{2m},
\end{equation*}
is obtained, where the possible wave vectors $k_n = n\cdot\frac{\pi}{a}$ are determined by periodic boundary conditions.
$a$ denotes the lattice parameter.
This means that the dispersion relation for free electrons inside the lattice is periodically parabolic.

The periodicity can be translated into the reduced zone scheme by mirroring.
However, at the edges of the first Brillouin zone, the electrons satisfy the Bragg condition and are therefore mirrored to the opposite side of the first Brillouin zone.
This leads to the creation of standing waves in k-space at both zone edges.
For a periodic potential generated by the nuclei in the lattice, the electron probability density is able to snap into the lowest energy solutions in two different ways:
\begin{itemize}
\item \textbf{Maxima of $|\psi|^2$ at locations of nuclei:} As a consequence of the tight Coulomb binding between the electrons and nuclei, the energy is lowered with respect to the free parabolic dispersion. The parabola is bending downwards.
	\item \textbf{Maxima of $|\psi|^2$ between locations of nuclei:} Illustratively speaking, the electrons are located further away from the nuclei. Hence, the energy is raised with respect to the free parabolic dispersion as a consequence of the looser Coulomb binding. The parabola is bending upwards.
\end{itemize}

In total, the parabola bends cause the band structure by introducing a \textit{band gap}.
Electronic states which lie inside the band gap are forbidden.
The resulting bands are now filled up, considering the Pauli exclusion principle.
At \SI{0}{\kelvin}, there are no electrons with energies above the Fermi energy $E_\text{F}$.
At room temperature, however, some electrons are thermally excited into states above the Fermi energy, since the Fermi distribution
\begin{equation*}
  \overline{n_\lambda} = \frac{1}{\e^{\beta\left(\epsilon_\lambda-\mu\right)} + 1}
\end{equation*}
is only approaching a Heaviside theta function for $T\rightarrow 0$.

Depending on the filling of these bands, a distinction is made between
\begin{itemize}
	\item \textbf{isolators:} valence band full, conduction band empty, large band gap,
	\item \textbf{semiconductors:} valence band full, conduction band empty, small band gap,
	\item \textbf{metals:} valence band full, conduction partially full.
\end{itemize}
