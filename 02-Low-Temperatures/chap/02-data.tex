\chapter{Data}

\section{Errors}
Since the lab manual does not specify any error boundaries on the used equipment, it is not possible to estimate any errors for the experiment in a reasonable manner.
Statistical errors from the performed fits are expected to be negligably small compared to the systematic errors, so they are omitted as well.

\section{Temperature Dependence of Resistance}\label{sec:samples}
Three samples are to be measured: a copper sample as a thin wire, a phosphorus-doped silicon sample and an evaporated layer of niobium.
Following behaviors are expected for these materials (as discussed in \autoref{chap:theory}):
\begin{itemize}
	\item \textbf{Copper sample} For the high temperature limit, the resistance should be linear in $T$, at about the debye temperature $\theta$, $R\propto T^5$ and for low temperatures, $R=\text{const.}$ since only the residual resistance caused by defects remains.
	\item \textbf{Silicon sample} Since phosphorus-doped silicon is a doped semiconductor, an inverse exponential dependence is expected, so at high temperatures the resistance ought to be very low, whereas for low temperatures the resistance should increase quickly.
	\item \textbf{Niobium sample} For high and intermediate (around debye temperature) temperatures, it should behave like the copper sample but at a certain temperature $T_\text{c}$, the resistance should vanish as the material transitions into a superconducting phase.
\end{itemize}

\subsection{Overview}
\begin{figure}
	\centering
	\begin{subfigure}{\textwidth}
		\centering
		\includegraphics[width=\linewidth]{./data/plots/overview.pdf}
		\caption{\textbf{Overview}}
	\end{subfigure}
	\hspace*{\fill}
	\begin{subfigure}{\textwidth}
		\centering
		\includegraphics[width=\linewidth]{./data/plots/overview-mag.pdf}
		\caption{\textbf{Magnified for low temperature regime}}
	\end{subfigure}
	\captiond{Resistance as a function of temperature}{for all three samples}
	\label{fig:overview}
\end{figure}

\autoref{fig:overview} depicts the temperature dependence of all three samples.
It is easy to see that, compared to the other samples, niobium has a rather high resistance, which decreases linearly until it reaches its critical temperature $T_\text{c}\approx\SI{5}{\kelvin}$, at which it almost instantly drops to \SI{0}{\ohm}.
As expected, copper also decreases linearly while the silicon sample shows a low resistance for high temperature with a sharp increase for low temperatures.
In the following, each sample will be assessed individually.

\subsection{Copper}
\begin{figure}
\begin{adjustwidth}{-1cm}{-1cm}
	\centering
	\begin{subfigure}{.55\textwidth}
		\centering
		\includegraphics[width=\linewidth, page=1]{./data/plots/separate.pdf}
		\caption{\textbf{Overview}}
	\end{subfigure}
	\hfill
	\begin{subfigure}{.55\textwidth}
		\centering
		\includegraphics[width=\linewidth, page=1]{./data/plots/separate-mag.pdf}
		\caption{\textbf{Magnified for low temperature regime}}
	\end{subfigure}
	\captiond{Resistance as a function of temperature}{for copper}
	\label{fig:copper}
\end{adjustwidth}
\end{figure}

\autoref{fig:copper} shows the temperature dependence of the copper sample, recorded for cooldown as well as warmup.

\section{Superconductor}
The temperature at which the sample becomes superconducting cannot be read off directly as the transition to zero resistance is not instantaneous, rather the steep slope between the residual resistance and zero resistance transitions into an asymptote to $R = 0$.
Thus, to get a repeatable measure of the transition temperature, the steep slope, which behaves approximately linearly, is extended to find it's intersection with $R = 0$, which is used as the temperature for each setting of the magnetic field.
\begin{figure}
	\centering
	\includegraphics[width=.6\textwidth]{./data/plots/Bc2-over-T.pdf}
	\captiond{Critical magnetic flux density $B_\text{C2}$ of Nb as a function of temperature}{The transition temperature of Nb is measured for magnetic fields up to \SI{0.6}{\tesla}}
	\label{fig:Bc2-over-T}
\end{figure}
The resulting data is shown in \autoref{fig:Bc2-over-T}.
A linear regression of form $B_\text{C2}(T) = a + b \cdot T$ is used to find the parameters $a = \SI{5.37}{\tesla}$ and $b = \SI{-0.630}{\tesla\per\kelvin}$.
Comparing the orders in $T$ between the formulae given in the experiment instructions and rearranging for the coherence length $\xi_\text{GL}(T = 0)$ and mean free path $l$ yields
\begin{alignat*}{2}
	\xi_\text{GL}(T = 0) &= \sqrt{\frac{\Psi_0}{2 \uppi a}} &= \SI{7.83e-9}{\meter}\\
	l &= \frac{\xi^2_\text{GL}(T = 0)}{\SI{39}{\nano\meter}} &= \SI{1.57e-09}{\meter}
\end{alignat*}
\todo{instructions say to only use the first three points (see page 2 of addenum)}

\section{Semiconductor}
Plotting the logarithm of the conductivity of the silicon sample over the reciprocal temperature yields a curve similar to the theoretical curve shown in \autoref{fig:itscomplicated}.
Area $\upbeta$ contains significantly more samples than area $\upalpha$, so only the samples in $\upbeta$ are used for the linear fit.
The fit of the linearized data yields a conductance of
\begin{equation*}
	\sigma_\text{Si} = \sigma_\infty \cdot \exp\left(- \frac{E_2}{2 k T}\right),
\end{equation*}
with an activation energy of $E_2 = \SI{6.6e-3}{\eV}$ and $\sigma_\infty = \SI{5.11}{S\per\milli\meter}$.
This activation energy is small compared to the band gap energy of $E_\text{G} \approx \SI{1.1}{\eV}$, which is desirable in electronic applications.

\begin{figure}
	\centering
	\includegraphics[width=.6\textwidth]{./data/plots/lns-over-inv-T.pdf}
	\captiond{Conductance of silicon sample over temperature}{Axes are scaled as $\log(\sigma)$ and $\frac{1}{T}$.}
	\label{fig:lns-over-inv-T}
\end{figure}
