\chapter{Procedure}
The cryostat used in this experiment is made up of a double-walled outer and inner glass dewar.
The inner dewar contains a sample chamber enclosed by a superconducting solenoid.
After evacuation of the inner dewar and its walls to clear out any residues, the walls are refilled with a small amount of helium and the inner dewar is flushed out with helium.
The outer dewar is filled with liquid nitrogen to cool down the samples (discussed in \autoref{sec:samples}).
At about \SIrange{80}{90}{\kelvin} liquid helium is poured into the inner dewar to speed up the cooling process.

The resistances of the temperature sensor and the three samples are displayed on four digital bench multimeters.
To ensure simultaneous readout of all values, the multimeters are recorded with a video camera and the video is paused to read off a set of values.
From around \SI{60}{\kelvin}, temperatures are recorded with a carbon resistor instead of the platinum thermometer since it features a greater resolution for this temperature range.
Resistances are measured for cooling down the samples as well as for warmup, where warming up the samples is accomplished by employing a PI-controlled heater to accurately sweep the temperature of the sample chamber.

To plot the resistance of the Nb sample against the resistance of the temperature sensor, an $x$-$y$ plotter is used.
This $T$-$R$ curve is recorded for magnetic fields from \SIrange{0}{0.6}{\tesla} in \SI{0.1}{\tesla} increments to determine the dependence of the critical temperature on the magnetic field.
