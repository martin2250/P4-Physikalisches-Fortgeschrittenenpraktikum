\chapter{Theory}
This experiment explores the electrical behavior of various conductors and semiconductors at cryogenic temperatures.

\section{Classical Electrical Conductivity}
The classical concept of free, localized electrons gives a simple explanation for electrical resistance, assuming that the electrons can scatter on the nuclei of the conductor material.
An external electric field $E$ accelerates the electrons with mass $m$, which regularly lose their kinetic energy to elastic collisions with nuclei.
The mean time between collisions $\tau$ depends on the parameters of the conductor's lattice structure.
Solving the equations of motion gives the term
\begin{equation*}
	\frac{m}{\tau} v_\text{D} = - e E
\end{equation*}
for the mean drift velocity $v_\text{D}$. The current density $j$ is given by $j = - e n v_\text{D}$, where $n$ denotes the density of electrons.
Using the definition of the conductance $j = \sigma E$, yields
\begin{equation}\label{eq:classical-resistance}
	\sigma = \frac{n e^2 \tau}{m}.
\end{equation}

\section{Temperature Dependence}
Most materials show a dependence on temperature in their electrical properties.
The density of free electrons $n$ in metals is independent of temperature, so the dependency must be in $\tau$, as the other variables in the RHS of \autoref{eq:classical-resistance} are constants.

In addition to the nuclei, electrons can also scatter on lattice defects and, more importantly, phonons, which are strongly associated with temperature.
The scattering processes are independent of each other, the total scattering rate $\tau^{-1}$ is calculated as the sum of the scattering rates due to phonons $\tau_\text{ph}^{-1}$ and nuclei/defects $\tau_\text{de}^{-1}$
\begin{equation*}
	\frac{1}{\tau} = \frac{1}{\tau_\text{ph}} + \frac{1}{\tau_\text{de}}.
\end{equation*}
Similarly, the specific resistance $\rho$ can be written as
\begin{equation*}
	\rho = \rho_\text{ph}(T) + \rho_\text{de},
\end{equation*}
the temperature independent $\rho_\text{de}$ is also called residual resistance, as it persists even at low temperatures, where phonon scattering is suppressed.
