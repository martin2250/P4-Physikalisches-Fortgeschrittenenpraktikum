\chapter{Theoretical Preliminaries}

\section{Two Dimensional Electron Gas (2DEG)}
\begin{figure}
	\centering
	\includegraphics[width=.5/textwidth]{./img/bandstructure-2DEG.pdf}
	\caption[Band structure of AlGaAs-GaAs heterostructure]{\textbf{Band structure of AlGaAs-GaAs heterostructure}}
	\label{fig:bandstructure_GaAs}
\end{figure}
In a two dimensional electron gas, the movement of electrons is confined to a plane, thus leading to quantized levels of motion in the third direction.
For the most problems, these levels can then be ignored.
One way of creating such a 2DEG is the employment of semiconductor-oxide-interfaces.
\autoref{fig:bandstructure_GaAs} shows the band structure of such a AlGaAs-GaAs-junction.
Since the band edges of the materials differ vastly, a thin triangular potential well is formed with its minimum below the Fermi level, effectively confining the electrons to a small region around the junction.
Hence, for low enough temperatures, only the lowest level of the quantum well is occupied and so the motion perpendicular to the junction surface can safely be ignored.
However, the electron still is free to move in a direction parallel to the junction, effectively realizing a 2DEG.

The dispersion relation of a 2DEG is
\begin{equation}\label{eq:subband}
	E_{k_\text{x}, k_\text{y}} = E_0 + \frac{\hbar^2}{2m_\text{eff}}\left(k_\text{x}^2 + k_\text{y}^2\right),
\end{equation}
where $E_0=E_\text{c} + E_\text{s}$ is the sum of the energy of the conduction band and the energy of the quantum well sub-band.

The density of states is calculated by counting the states in k-space
\begin{equation*}
	N(\epsilon) = \underbrace{2}{spin degeneracy}\frac{\pi k^2\cdot S}{4\pi^2} = S\frac{2m_\text{eff}\epsilon}{2\pi\hbar^2} = S\frac{m_\text{eff}\epsilon}{\pi\hbar^2}.
\end{equation*}
and differentiating with respect to energy
\begin{equation*}
	\nu(\epsilon) = \frac{1}{S}\cdot\diff{N(\epsilon)}{\epsilon}\vartheta(\epsilon-E_0) = \frac{m_\text{eff}}{\pi\hbar^2} = \text{const.}.
\end{equation*}

\section{2DEG In a Magnetic Field}
We want to solve the stationary Schrödinger Equation
\begin{equation*}
	\frac{1}{2m_\text{eff}}\left(\vec{p}-e\vec{a}\right)^2\psi = E\psi.
\end{equation*}
In Lorentz gauge, we can choose $\vec{A}=(0, Bx, 0)^T$ and solve by a product ansatz $\psi(x,y,z) = \phi(x,y)Z(z)$.
The z-equation is readily solved with the eigenenergies being the sub-band energies mentioned in \autoref{eq:subband}.
Solving the eigenequation for $\phi(x,y)$ yields the \textit{Landau levels}
\begin{equation*}
	\epsilon_n = \hbar\omega_\text{c}\left(n+\frac{1}{2}\right)\qquad{n\in\mathbb{N}_0},
\end{equation*}
where $\omega_\text{c} = \frac{eB}{m}$ is called the cyclotron frequency.
